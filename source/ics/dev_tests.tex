\documentclass[11pt]{article}

\usepackage{natbib}
\bibpunct{(}{)}{;}{a}{,}{,}

\newcommand{\bq}{\begin{equation}}
\newcommand{\eq}{\end{equation}}



\setlength{\oddsidemargin}{-1.0cm}
\setlength{\textwidth}{18.0cm}
\setlength{\topmargin}{-1.0cm}
\setlength{\textheight}{25.0cm}
%\parskip = 6.0 truept
\parskip = 0.1cm
%\parindent0pt

\begin{document}

\title{Test Problems For My Code, and Current Status}
\author{Jonathan Mackey}
\maketitle

\section{Introduction}
This is for my uniform grid code.  Current features:
\begin{itemize}
\item Solves the following equations: Euler, Ideal MHD, Ideal MHD with
Dedner \textit{mixed-GLM} divergence cleaning, and advection of
passive tracers.
\item Can be run in serial, or in parallel via MPI.
\item Data I/O as ASCII text, FITS, and Silo data, with FITS and Silo
also working in parallel.
\item Module for Dynamics is fully parallel for all the equations.
\item Module for Raytracing can handle parallel rays in serial and
parallel mode, and point sources in serial mode only.
\item Microphysics module doesn't know about serial or parallel, as it
works entirely within a cell.  It can integrate rate equations, but it
a bit rough around the edges.  Photo-ionisation of hydrogen is fairly
robust at this stage.
\end{itemize}

\section{Test Problems}
\vspace{0.4cm} \hrule
\subsection{Uniform Initial Conditions, with Noise}
This basically just tests the stability of the algorithms, and that
the parallel version doesn't change the results at all.  If the
algorithm is unstable, even a small level of noise will blow up into
unphysical conditions very quickly.
\subsubsection{Status}
\textit{2008-09-10} This is working for text, fits, and silo I/O, in
serial and parallel.  In parallel I have changed the seeding of the
noise so that different processors have a different seed -- they all
have the same basic seed, which I add the processors rank to, so the
seed can be recovered.

\vspace{0.4cm} \hrule
\subsection{Double Mach Reflection}
\textbf{Ref?} This tests the quality of the algorithms, as well as
solving quite a difficult problem.  If the reflective, inflow, and
outflow boundary conditions have problems, it will show up in this
test.  This generates a number of shock waves and discontinuities,
some of which should be straight lines, and which are at different
angles to the grid axes.  If the algorithm has insufficient artificial
viscosity, the shocks will not stay as straight lines, and in
particular there will be problems near the reflection axis.
\subsubsection{Status}
\textit{2008-09-10} This is working for fits and silo I/O in serial
and parallel.  It solves the Euler Equations and my results look great
as far as I can tell.  If I use Sam's A-V prescription, it works fine.

\vspace{0.4cm} \hrule
\subsection{Advection of Dense Clump}
This has a dense clump advected across a grid at a fixed velocity,
with periodic boundary conditions, so it basically tests how well the
code can handle contact discontinuities, and how well it can handle a
moving grid.  Grid codes are very bad at this historically, and the
(necessary) dissipation eventually spreads the clump out.
\subsubsection{Status}
\textit{2008-09-11} Works in serial mode with fits/silo I/O in 3D.
Also in parallel in 3D.  Am assuming it works in 2D too.  This is
glm-mhd equations testing, with one colour tracer.

\vspace{0.4cm} \hrule
\subsection{Orszag-Tang Vortex}
This problem tests mode MHD algorithms to their limits.  A periodic 2D
domain is set up with trans-sonic velocities.  Shock waves and
vortices quickly form. from \textit{Dai \& Woodward 1998, ApJ, 494,
317.}
\subsubsection{Status}
\textit{2008-09-11} Works fine in serial and parallel with silo, and
previously with fits so no reason to think it is broken now.  One
point is that Jim Stone's code test page has a movie that shows a
strong vortex developing in the middle, and I don't get that, I think
because I have added viscosity.  But without viscosity it bugs out and
the results aren't reliable anyway.  I ran it today with
$\eta_v=0.05$, and it is chugging away happily.

\vspace{0.4cm} \hrule
\subsection{Hydro Kelvin-Helmholtz Instability}
This can be set up in a number of ways: e.g.\ Jim Stone's test has
periodic boundaries on all sides with different density fluids,
whereas most others have same density fluids and reflective boundaries
at the top and bottom boundaries (Frank et al., 1996, ApJ, 460, 777).
I decided to do both setups in the ic generator, but am not too sure
which I will use in the test suite.  For the Euler equations I'd say
either is fine, except that Frank's situation leads to the `cat's eye'
vortex forming at the interface, which is a stable final
configuration, so that's nice to have.
\subsubsection{Status}
\textit{2008-10-13} Am going with Frank's hydro setup -- constant
density and pressure over the whole initial state, periodic
horizontally and reflecting vertically, with the upper boundary have
velocity $U_x$ and the lower boundary an equal and opposite boundary.
There is a shear boundary layer of thickness $L_y/25$, specified with
a $\tanh()$ function.  This is run with $\eta_v=0.15$, finish time
$=15$, CFL$=0.4$, $\gamma=1.4$, and a $128\times128$ grid.  The
perturbations are in $v_y$ and are the sum of three sine waves with 1,
2, and 15 wavelengths on the $x$-axis, damped with a Gaussian filter
in the $y$-direction:
\[ v_y = \frac{U_y}{3}\left( \sin(30\pi x/L_x) + \sin(4\pi x/L_x) +
  \sin(2\pi x/L_x) \right)\,. \]
I experimented and found the results don't depend much on the
perturbations imposed, but in some cases you can get two vortices
whereas in others they will merge eventually into a single vortex.

This problem is interesting for seeing how artificial viscosity
affects the solution.  For example switching off viscosity altogether
produces entirely different results to having $\eta_v=0.005$, which is
a \emph{very} small value.  for $128^2$ simulations, $\eta_v=0.15$
severely damps the perturbations on all but the largest scales.
Stone's test is interesting in that is is easy to see the
perturbations start on small scales and grow as the vortices merge, so
it might be better to use it now that I think about it.  Or else to
use both for a while and see which one I like better in future.
Stone's setup is periodic box of unit area centred on the origin, with
$[\rho,p_g,v_x]=[2,2.5,0.5]$ for $|y|<0.25$ and $[1,2.5,-0.5]$ for
$|y|>0.25$, yielding two slip surfaces.  The shear layer is sharp.  He
uses $\gamma=1.4$ and a random perturbation in $v_y$ with min-max
amplitude of 0.01.  He runs it to a time $t=5$.


\vspace{0.4cm} \hrule
\subsection{MHD Kelvin-Helmholtz Instability}
This is not as clean a test as the Hydro KHI, because it doesn't
settle down to a stable end configuration.  Instead it settles to a
quasi-steady sequence of rolling waves followed by tearing and
reconnection of field lines, contained within a broad `shear layer.'
The extent to which this happens depends on the field strength, with
strong fields inhibiting the instability altogether.  Even very weak
fields, however, don't reduce to the hydrodynamic limit because the
vortices twist up the field until it becomes locally strong.  I am
using Frank's setup for this too.  This has the same initial
conditions as the hydro test above, but with a uniform magnetic field
of strength $B_x=0.2$ or $0.4$ throughout the grid (note not reversed
with the flow reversal).  Stone's version of the test is the same as
his hydro test, but with $B_x=0.5$ (but this may be divided or
multiplied by $\sqrt{4\pi}$, I can't tell, but from comparing my
results to his it must be $0.5/\sqrt{4\pi}$; using 0.5 has too strong
a field and the instability is damped).
\subsubsection{Status}
\textit{2008-10-13} Still working on this.  I think a $256^2$ model
will work for Frank's weak field test, but the results are so
sensitive to everything that a byte-for-byte file comparison won't
really work.  I'll need to use it as a test only for visual inspection
of the stages of growth.  The growth rate is even very resolution
dependent from looking at Stone's webpage and Frank's figures.  I
might want to look at Stone's test for this too, since it is again
easier to visualise what it going on.


\vspace{0.4cm} \hrule
\subsection{}
\subsubsection{Status}
\textit{2008-09-11}

\section{Non-adiabatic tests (with cooling and/or microphysics)}
\vspace{0.4cm} \hrule
\subsection{Radiative Shock Test}
This is based on John Raymond's (1979) shock test problems, but I am
doing them on a 1D grid and explicitly following the dynamics whereas
he imposed shock jump conditions and let the post-shock gas evolve
isobarically.  My results are very comparable to his.
\subsubsection{Status}
\textit{2008-10-13} Have a stable setup, just need to write it up.

\vspace{0.4cm} \hrule
\subsection{}
\subsubsection{Status}
\textit{2008-09-11}

\section{Still to Do}
\begin{itemize}
\item divB peak,
\item radiative shock test -- (1d,2d,w/cooling),
\item blast wave -- (2d,3d,centred and reflecting,axisymmetric,mhd/euler),
\item shock-cloud -- (2d(cart/cyl) and 3d, euler/mhd, w/cooling?)
  shock-tubes!
\item Jets -- 2d(cart/cyl) and 3d, euler/mhd, w/cooling, w/atomics.
\item Ifront-cloud -- 2d(cart/cyl) and 3d, euler and mhd, w/cooling,
  parallel/pt.src.
\item Str\"omgren Sphere: 2d(cart/cyl) and 3d, euler/mhd, w/cooling.
\end{itemize}


%\subsection{}
%\subsubsection{Status}
%\textit{2008-09-xx}

\end{document}
